\chapter{Conclusiones y trabajos futuros.}
\epigraphhead [30]{%
  \epigraph{``Ningún conocimiento humano puede ir más allá de su experiencia.''}%
  {John Locke, filósofo.}%
}

\section{Conclusiones personales.}
	Desde el principio he querido realizar un proyecto relacionado de alguna manera con una Raspberry Pi o un SBC con el objetivo de demostrar que muchas de las soluciones informáticas que hay en el mundo real en distintos ámbitos, especialmente en sistemas embebidos, son excesivas y pueden realizarse con presupuestos muy inferiores. Esto en parte es posible gracias al \emph{Open Source} tanto referido al software como al hardware y la Raspberry Pi es totalmente libre en los dos ámbitos.
	
	Este proyecto ha demostrado que efectivamente es posible abaratar costes y realizar soluciones viables mediante el uso de estos ordenadores de bajo coste, además de suponerme un desafío en general. No solo porque es el proyecto de fin de carrera donde uno ha de poner en práctica todos los conocimientos adquiridos durante estos años, sino porque además he tenido que convivir con otro proyecto paralelo con el que se perseguía un objetivo común. 
	
	Ésto ha provocado un montón de acondicionamiento e incertidumbre durante todo el proceso. Pero como en todo trabajo en equipo, la buena comunicación y la ayuda mutua han cumplido su cometido y hemos podido cumplir no todos, pero gran parte de los objetivos cuando en las primeras reuniones no se tenía claro si realmente esta solución iba a ser viable.
	
	Es otra de las razones que han hecho de este proyecto un desafío: No se sabía exactamente si la plataforma iba a ser capaz de sustituir a un equipo mucho más caro, lo cual lo ha provocado que una buena parte del mismo haya sido un estudio de viabilidad e investigación. 
	
	Personalmente estoy contento con el resultado. Si bien es un trabajo que puede llegar mucho más lejos, se ha demostrado lo que quería probar en primer lugar, y en el proceso he podido aprender un método de trabajo nuevo, un lenguaje de programación importante como Python, y además en la realización de esta memoria también he conocido \LaTeX  y gran parte de su potencia.
	
	
\section{Conclusiones profesionales.}
	El trabajo colaborativo ha sido sin duda el pilar más importante de este proyecto, puesto que de otra manera no habría sido posible llegar a donde se ha llegado. Ambos proyectandos provenimos de distintas carreras, yo por una parte de Ingeniería Informática y mi compañero de Ingeniería de Telecomunicaciones. Cada uno se ha centrado en su campo de estudio y ha sacado su parte adelante respetando la opuesta para que al final la unión de ambas haya sido un éxito.
	
	
\section{Trabajos futuros.}
	\subsection{Nuevas funcionalidades..}
		Además de las funcionalidades que no se han desarrollado por no ser un objetivo del proyecto pero se han propuesto y en ocasiones preparado a falta de escribir el modelo como la exportación PDF o la utilización del mismo en otros sistemas operativos, este programa puede albergar otras funcionalidades como por ejempo la obtención de estadísticos con los datos obtenidos.
		
	\subsection{Optimización de la solución.}
		Aunque el rendimiento actual es aceptable y muy válido, siempre se puede optimizar por ejemplo utilizando una versión del sistema operativo para tiempo real o modificando éste para que la integración en el mismo sea total, recortando programas y servicios que no resulten útiles para el propósito de este programa.
		
		También es posible conseguir una mejora a través de un estudio de la optimización del diseño de la aplicación o portando la solución a lenguajes más eficientes como puede ser C.
		
	\subsection{Adaptación a otras SBCs.}
	La Raspberry Pi es realmente la SBC menos potente de todas, sería interesante hacer el programa compatible con su clon de doble núcleo la Banana Pi. 
	
	Se dispuso de una unidad durante el desarrollo y se comprobó que el funcionamiento el programa y todo el sistema en general era muchísimo más ligero. Solo se pudo probar el funcionamiento con el simulador de datos ya que la librería piDA no está adaptada a trabajar con una Banana Pi, pero una vez realizada esa adaptación sería interesante ver de qué es capaz esta placa con toda su potencia y tan solo 20€ más de sobreprecio con respecto a la Raspberry Pi. Es muy probable que resulte en una solución más viable.
	
	\subsection{Integración del producto final en los laboratorios de la Facultad.}
	La solución funcional podría integrarse a pequeña escala en los laboratorios de la Facultad con el fin de que los alumnos hagan uso de ella. Se puede evaluar así la utilidad de la misma y la calidad en el propio entorno para el que ha sido diseñada, obteniendo así comentarios y opiniones del usuario final muy útiles.
	
	\subsection{Explotación comercial.}
	Sí la integración en los laboratorios de la Facultad es un éxito, se puede dar el paso a crear un producto comercial. La explotación del mismo empezaría por una producción industrial de la interfaz de adquisición de datos y su venta vía web. 
	
	Si realmente es un producto funcional y supone un ahorro tan grande con respecto a las soluciones de la competencia puede resultar una fuente de ingresos que remunere el esfuerzo y la idea que ha habido tras el mismo.