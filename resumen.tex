\chapter*{Resumen}
Actualmente algunos laboratorios de Física de la Facultad de Ciencias tienen problemas de mantenimiento, bien porque se necesita instrumental obsoleto y en extinción por cuestiones de compatibilidad con elementos de difícil y costoso reemplazo o bien porque su coste es relativamente alto como para ser sustituido. Actualmente una manera de abordar este tipo de problemas en con SBC (single board computer). El más destacado miembro, junto con Arduino, de este nuevo club de ordenadores de placa única es el Raspberry Pi.

En este PFC se analizan las necesidades de estos laboratorios, se propone y diseña una solución software basada en Raspberry Pi apoyada en un hardware creado en un proyecto paralelo y se realizan las pruebas y evaluaciones necesarias así como un análisis de los requisitos y expectativas de explotar la solución de forma industrial.

\section*{Palabras clave.}
Single board computer, SBC, Raspberry Pi, Python, Linux, ARM.

\chapter*{Abstract}

Some laboratories in the Physics department at the Facultad de Ciencias have nowadays maintenance problems 
due to old and obsolete equipment, which is difficult to replace because of its high cost. One of the many ways to 
face this kind of problems is with the use of SBCs (single-board computers), the best known amongst them being the Raspberry Pi.

In this End of Degree project we analyse the needs of these laboratories, propose and design a software solution
based on the Raspberry Pi and supported by a hardware created in a parallel project. 
The solution is tested and evaluated altogether, and analysed for potential commercial exploitation.

\section*{Keywords.}
Single board computer, SBC, Raspberry Pi, Python, Linux, ARM.